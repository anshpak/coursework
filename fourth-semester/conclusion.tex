\chapter*{ \large ЗАКЛЮЧЕНИЕ}
\addcontentsline{toc}{chapter}{ЗАКЛЮЧЕНИЕ}
В работе были даны определения скалярного уравнения Риккати, а так же уравнений Риккати высших порядков и матричного уравнения Риккати. Были выведены уравнения в $\mathbb{R}_N$, $\mathbb{R}_3$ и $\mathbb{R}_2$. Также были проанализированы основные свойства решений скалярного уравнения Риккати.

Было установлено, что свойства сохранения вида уравнения Риккати при любом преобразовании независимой переменной и при любом дробно-линейном преобразовании выполняются и для уравнений второго и третьего порядка. При этом свойства, указанные для скалярного уравнения Риккати оказываются справедливы не только для уравнений высших порядков, но и для матричного уравнения Риккати, в том числе и сведение к линейному уравнению, а так же результат отношения четырех частных решений уравнения.