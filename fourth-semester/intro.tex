\chapter*{\large ВВЕДЕНИЕ}  
\addcontentsline{toc}{chapter}{ВВЕДЕНИЕ}
Дифференциальные  уравнения --- раздел  математики, изучающий теорию и способы решения уравнений, содержащих искомую функцию и ее производные различных порядков одного аргумента (обыкновенные дифференциальные) или нескольких аргументов (дифференциальные уравнения в частных производных).В самом уравнении участвует не только неизвестная функция, но и различные ее производные. Дифференциальным уравнением описывается связь между неизвестной функцией и ее производными. Такие связи отыскиваются в различных областях знаний: в механике, физике, химии, биологии, экономике и др.

Дифференциальные уравнения применяются для математическогоописанияприродных явлений. Так, например, в биологии дифференциальные уравнения применяются для описания популяции; в физикемногие законы можно описать с помощью дифференциальных уравнений. Широкое применение находят дифференциальные  уравнения и в моделях экономической динамики. В данных моделях отражается не только зависимость переменных от времени, но и их взаимосвязь во времени.

В качестве объекта исследования в данной курсовой работе взяты скалярные уравнения Риккати и уравнения Риккати высших порядков, а предметом исследования являются их свойства.

Целью данной работы является характеристика основных свойст скалярного уравнения Риккати и проверка их справедливости для уравнений Риккати второго и третьего порядков.

Были поставлены задачи:
\begin{enumerate} 
  \item Дать определение скалярного уравнения Риккати.
  \item Проанализировать основные свойства решений скалярного уравнения Риккати.
  \item Дать определения уравнений Риккати высших порядков.
  \item Проверить выполнение основных свойств решений скалярного уравнения Риккати для уравнений Риккати высших порядков.
  \item Дать определения матричного уравнения Риккати.
  \item Охарактеризовать основные свойства решений матричного уравнения Риккати.
\end{enumerate}