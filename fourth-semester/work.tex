\chapter{Уравнение Риккати, матричные уравнения Риккати, уравнения Риккати высших порядкой. Основные свойства решений скалярного уравнения Риккати} 

\section{Существование и единственность решения задачи Коши.}
Рассмотрим уравнение:

$$\dfrac{dy}{dx} = f \left( x, y \right),$$

в котором правая часть есть квадратичная функция (от искомой функции) $y$, т. е.
\begin{equation}  \label{eq:one_one}
    \dfrac{dy}{dx} = P \left( x \right) y^2 + Q \left( x \right) y + R \left( x \right).
\end{equation}

Такое уравнение называется уравнением Риккати. Будем считать, что функции $P \left( x \right)$, $Q \left( x \right)$, $R \left( x \right)$ определены и непрерывны в интервале $\left( a, b \right) \; \left( a \geqslant -\infty,\right.$ $\left.( b \leqslant +\infty \right),$ причем $P \left( x \right) \not\equiv 0$ и $R \left( x \right) \not\equiv 0$ в этом интервале (в противном случае уравнение Риккати вырождается в линейное уравнение или уравнение Бернулли).

При сделанных предположениях относительно $P \left( x \right)$, $Q \left( x \right)$ и $R \left( x \right)$ уравнение Риккати \ref{eq:one_one} имеет единственное решение
\begin{equation}  \label{eq:one_two}
    y = y \left( x \right)
\end{equation}
удовлетворяющее начальному условию:
\begin{equation}  \label{eq:one_three}
    y = y_{0} \; \; \text{при} \; \; x = x_{0},
\end{equation}
где $x_{0}$ принадлежит интервалу $\left( a, b \right),$ а за $y_{0}$ можно брать любое число, т. е. через каждую точку $\left( x_{0}, y_{0} \right)$ полосы

\begin{equation}  \label{eq:one_four}
    a < x < b, \; \; -\infty < y < +\infty
\end{equation}
проходит одна и только одна интегральная кривая уравнения Риккати.

Действительно, всегда можно построить прямоугольник
$$ R: \; \; \left| x - x_{0} \right| \leqslant a_{1}, \; \; \left| y - y_{0} \right| \leqslant b_{1}$$
с центром в точке $\left( x_{0}, y_{0} \right),$ который целиком лежит в полосе $\left( \ref{eq:one_four} \right)$. В этом прямоугольнике правая часть уравнения Риккати $\left( \ref{eq:one_one} \right)$ удовлетворяет обоим условиям теоремы Пикара. А тогда уравнение $\left( \ref{eq:one_one} \right)$ имеет единственное решение $\left( \ref{eq:one_two} \right)$, удовлетворяющее начальному условию $\left( \ref{eq:one_three} \right)$. Это решение определено вообще говоря, лишь в некоторой окрестности точки $x = x_{0}.$ Существование этого решения во всем интервале непрерывности коэффициентов $P \left( x \right)$, $Q \left( x \right)$ и $R \left( x \right)$ не гарантируется.

Пример. Рассмотрим уравнение
$$y' = y^{2} - 2 y + 1.$$

Здесь правая часть определена и непрерывна на всей плоскости $\left( x, y \right)$. Но из формулы общего решения
$$y = 1 - \dfrac{1}{x - C}$$
видно, что никакое из решений, входящих в эту формулу при $C \neq \infty,$ не будет определено при всех $x$.

Из сказанного выше следует, что уравнение Риккати не имеет особых решений. Всякое решение его есть частное решение.

\section{Общие свойства уравнения Риккати.}%
$1.$ Уравнение Риккати, так же, как и линейное уравнение, сохраняет свой вид при любом преобразовании независимой переменной

\begin{equation}  \label{eq:one_five}
    x = \varphi \left( t \right),
\end{equation}
где $\varphi \left( t \right)$ --- любая непрерывно-дифференцируемая функция, определенная в интервале $\left( t_{0}, t_{1} \right),$ причем $\varphi' \left( t \right) \neq 0$ в $\left( t_{0}, t_{1} \right).$

Действительно, так как 
$$\dfrac{dy}{dt} = \dfrac{dy}{dx} \varphi'\left( t \right),$$
то преобразованное уравнение имеет вид:
$$\dfrac{dy}{dt} = \left\{ P \left[\varphi\left( t \right) \right] y^{2} + Q \left[\varphi\left( t \right) \right] y + R\left[\varphi\left( t \right) \right] \right\} \phi' \left( t \right),$$
т. е. является опять уравнением Риккати.

$2.$ В отличие от линейного уравнения, уравнение Риккати сохраняет свой вид не только при любом линейном преобразовании искомой функции, но также и при любом дробно-линейном преобразовании
\begin{equation}  \label{eq:one_six}
    y = \dfrac{\alpha \left( x \right) z + \beta \left( x \right)}{\gamma \left( x \right) z + \delta \left( x \right),}
\end{equation}
где $\alpha \left( x \right), \beta \left( x \right), \gamma \left( x \right), \delta \left( x \right)$ ~--- произвольные функции, определенные и непрерывно дифференцируемые в интервале $\left( a, b \right)$, подчиненные лишь очевидному условию $\alpha \left( x \right) \delta \left( x \right) - \beta \left( x \right) \gamma \left( x \right).$

В самом деле, дифференцируя $\left( \ref{eq:one_six} \right)$, находим:
$$ 
    y' = \dfrac{\left( \alpha'z + \alpha z' + \beta' \right) \left( \gamma z + \delta \right) - \left( \alpha z + \beta \right) \left( \gamma' z + \gamma z' + \delta \right)}{\left( \gamma z + \delta \right)^2} =
$$
\begin{equation}  \label{eq:one_seven}
    = \dfrac{\left( \alpha \delta - \beta \gamma \right)z' + \left( \alpha' \gamma - \alpha \gamma' \right)z^{2} + \left( \alpha' \delta' + \beta' \gamma - \alpha \delta' - \beta \gamma' \right)z + \beta' \delta - \beta \delta'}{\left( \gamma z + \delta \right)^2},
\end{equation}



так что левая часть уравнения $\left( \ref{eq:one_one} \right)$ заменится дробью $ \left( \ref{eq:one_seven}. \right)$ Правая же часть уравнения $\left( \ref{eq:one_one} \right)$ после замены $y$ выражением $\left( \ref{eq:one_six} \right)$ и приведения к общему знаменателю обратится в дробь, числитель которой есть квадратичная функция от $z,$ а знаменатель --- тот же, что и у дроби $\left( \ref{eq:one_seven} \right)$. Поэтому преобразованное уравнение будет опять уравнением Риккати.

Применяя то или иное из указанных преобразований, можно значительно упростить вид уравнения Риккати и, таким образом, облегчить его изучение.
%\chapter{Определения  матричных уравнений Риккати и уравнений Риккати высших порядков}%
%\section{Тут может быть заголовок 1.}%

%\section{Тут может быть заголовок 2.}%

%\chapter{Основные свойства матричных уравнений Риккати и уравнений Риккати высших порядков из цепочки.}%

%\section{Тут может быть заголовок 1.}%

%\section{Тут может быть заголовок 2.}%

%\chapter{Анализ и сравнение основных свойств решений скалярного уравнения Риккати и матричных уравнений Риккати и уравнений Риккати высших порядков из цепочки.}%
%\section{Тут может быть заголовок 1.}%

%\section{Тут может быть заголовок 2.}%