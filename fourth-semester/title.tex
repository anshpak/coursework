\setcounter{page}{1}
\setstretch{1.0}
\thispagestyle{empty}
\newgeometry{
	left=0mm,
    top=20mm,
    right=0mm,
    bottom=20mm
}
\begin{center}
\bf
\vspace{4cm}
{
\setstretch{0.9}
\mbox{МИНИСТЕРСТВО~ОБРАЗОВАНИЯ~РЕСПУБЛИКИ~БЕЛАРУСЬ} \\~\\
\mbox{БЕЛОРУССКИЙ~ГОСУДАРСТВЕННЫЙ~УНИВЕРСИТЕТ} \\~\\
\mbox{МЕХАНИКО-МАТЕМАТИЧЕСКИЙ~ФАКУЛЬТЕТ} \\~\\
\mbox{Кафедра~дифференциальных~уравнений~и~системного~анализа} \\~\\
}
\vspace{4cm}
\bf
\mbox{О СВОЙСТВАХ РЕШЕНИЯ УРАВНЕНИЙ РИККАТИ ВЫСШИХ ПОРЯДКОВ}\\
\vspace{1cm}
\rm Курсовая работа 
\vspace{3cm}
\end{center}
\begin{tabular}{ll}
\hspace{10.5cm}
&Шпака Андрея Валерьевича~\\
&студента 2-го курса\\
&специальности 1-31 03 09\\
&<<Компьютерная математика\\
&и системный анализ>>\\~\\
&Научный руководитель:\\
&профессор, доктор физ.-мат. наук\\
&В.~И.~Громак
\end{tabular}
\vspace{3cm}
\begin{center}
Mинск, 2023
\end{center}
\clearpage
\restoregeometry