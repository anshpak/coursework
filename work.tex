\chapter{Основные свойства решений скалярного уравнения Риккати} 

\section{Существование и единственность решения задачи Коши.}
Рассмотрим уравнение:

$$\dfrac{dy}{dx} = f \left( x, y \right),$$

в котором правая часть есть квадратичная функция (от искомой функции) $y$, т. е.
\begin{equation}  \label{eq:one_one}
    \dfrac{dy}{dx} = P \left( x \right) y^2 + Q \left( x \right) y + R \left( x \right).
\end{equation}

Такое уравнение называется уравнением Риккати. Будем считать, что функции $P \left( x \right)$, $Q \left( x \right)$, $R \left( x \right)$ определены и непрерывны в интервале $\left( a, b \right) \; \left( a \geqslant -\infty,\right.$ $\left.( b \leqslant +\infty \right),$ причем $P \left( x \right) \not\equiv 0$ и $R \left( x \right) \not\equiv 0$ в этом интервале (в противном случае уравнение Риккати вырождается в линейное уравнение или уравнение Бернулли).

При сделанных предположениях относительно $P \left( x \right)$, $Q \left( x \right)$ и $R \left( x \right)$ уравнение Риккати \ref{eq:one_one} имеет единственное решение
\begin{equation}  \label{eq:one_two}
    y = y \left( x \right)
\end{equation}
удовлетворяющее начальному условию:
\begin{equation}  \label{eq:one_three}
    y = y_{0} \; \; \text{при} \; \; x = x_{0},
\end{equation}
где $x_{0}$ принадлежит интервалу $\left( a, b \right),$ а за $y_{0}$ можно брать любое число, т. е. через каждую точку $\left( x_{0}, y_{0} \right)$ полосы

\begin{equation}  \label{eq:one_four}
    a < x < b, \; \; -\infty < y < +\infty
\end{equation}
проходит одна и только одна интегральная кривая уравнения Риккати.

Действительно, всегда можно построить прямоугольник
$$ R: \; \; \left| x - x_{0} \right| \leqslant a_{1}, \; \; \left| y - y_{0} \right| \leqslant b_{1}$$
с центром в точке $\left( x_{0}, y_{0} \right),$ который целиком лежит в полосе \ref{eq:one_four}. В этом прямоугольнике правая часть уравнения Риккати \ref{eq:one_one} удовлетворяет обоим условиям теоремы Пикара. А тогда уравнение \ref{eq:one_one} имеет единственное решение \ref{eq:one_two}, удовлетворяющее начальному условию \ref{eq:one_three}. Это решение определено вообще говоря, лишь в некоторой окрестности точки $x = x_{0}.$ Существование этого решения во всем интервале непрерывности коэффициентов $P \left( x \right)$, $Q \left( x \right)$ и $R \left( x \right)$ не гарантируется.

Пример. Рассмотрим уравнение
$$y' = y^{2} - 2 y + 1.$$

Здесь правая часть определена и непрерывна на всей плоскости $\left( x, y \right)$. Но из формулы общего решения
$$y = 1 - \dfrac{1}{x - C}$$
видно, что никакое из решений, входящих в эту формулу при $C \neq \infty,$ не будет определено при всех $x$.

Из сказанного выше следует, что уравнение Риккати не имеет особых решений. Всякое решение его есть частное решение.

%\section{Тут может быть заголовок 2.}%

%\chapter{Определения  матричных уравнений Риккати и уравнений Риккати высших порядков}%
%\section{Тут может быть заголовок 1.}%

%\section{Тут может быть заголовок 2.}%

%\chapter{Основные свойства матричных уравнений Риккати и уравнений Риккати высших порядков из цепочки.}%

%\section{Тут может быть заголовок 1.}%

%\section{Тут может быть заголовок 2.}%

%\chapter{Анализ и сравнение основных свойств решений скалярного уравнения Риккати и матричных уравнений Риккати и уравнений Риккати высших порядков из цепочки.}%
%\section{Тут может быть заголовок 1.}%

%\section{Тут может быть заголовок 2.}%